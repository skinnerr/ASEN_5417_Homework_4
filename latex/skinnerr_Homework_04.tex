\documentclass[11pt]{article}

\input{../../Latex_Common/skinnerr_latex_preamble_asen5417.tex}

%%
%% DOCUMENT START
%%

\begin{document}

\pagestyle{fancyplain}
\lhead{}
\chead{}
\rhead{}
\lfoot{\hrule ASEN 5417: Homework 4}
\cfoot{\hrule \thepage}
\rfoot{\hrule Ryan Skinner}

\noindent
{\Large Homework 4}
\hfill
{\large Ryan Skinner}
\\[0.5ex]
{\large ASEN 5417: Numerical Methods}
\hfill
{\large Due 2015/10/13}\\
\hrule
\vspace{6pt}

%%%%%%%%%%%%%%%%%%%%%%%%%%%%%%%%%%%%%%%%%%%%%%%%%
%%%%%%%%%%%%%%%%%%%%%%%%%%%%%%%%%%%%%%%%%%%%%%%%%
\section{Introduction} %%%%%%%%%%%%%%%%%%%%%%%%%%
%%%%%%%%%%%%%%%%%%%%%%%%%%%%%%%%%%%%%%%%%%%%%%%%%
%%%%%%%%%%%%%%%%%%%%%%%%%%%%%%%%%%%%%%%%%%%%%%%%%

We solve the following problems to better understand numerical techniques for solving boundary value problems. As will be described in the methods section, our tools primarily consist of the Thomas algorithm and LU-decomposition.

\subsection{Problem 1}

Recall the boundary value problem for free convection along a vertical plate, as used in Problem 2 from Homework 3:
\begin{equation}
\begin{aligned}
F''' + 3 F F'' - 2F'^2 + \theta &= 0 \;, \\
\theta'' + 3 \text{Pr} F \theta' &= 0 \;,
\end{aligned}
\label{eq:prob1_original}
\end{equation}
\begin{equation}
\begin{aligned}
\eta = 0 &: &\quad F = F' =\; &0, &\quad \theta =\; &1 \;, \\
\eta \rightarrow \infty &: &\quad F' \rightarrow\; &0, &\quad \theta \rightarrow\; &0
\;.
\end{aligned}
\label{eq:prob1_original_bcs}
\end{equation}

Convert the 3\rd-order $F$-equation into a system of one 1\st- and one 2\nd-order equations, and solve this system coupled with the $\theta$-equation. With this approach, the BCs can be directly applied at $\eta = \{0, 10\}$. Solve the 2\nd-order equation with 2\nd-order central differences, and solve the 1\st-order equation with with the explicit Euler method. Use $N = 101$ equally-spaced grid points, and solve the three equations iteratively, subject to your own convergence criterion. Solve the resulting finite difference system for the 2\nd-order equations using the Thomas algorithm.
\begin{enumerate}
\item Compare results for $F'$ and $\theta$ to those obtained in Homework 3 for $\text{Pr} = \{1, 10\}$.
\item Note that this time, we are solving a coupled BVP; compare the gradients $F''(0)$ and $\theta'(0)$ for each Prandtl number with those from Homework 3.
\end{enumerate}

\subsection{Problem 2}

Discretize the boundary value problem
\begin{equation}
\frac{d^2 y}{d \theta^2} + \frac{y}{4} = 0
\;,\qquad
y(-1) = 0
\;,\qquad
y(1) = 2
\;,\qquad
-1 \le \theta \le 1
\label{eq:prob2_ode}
\end{equation}
using second-order central differences with $N = 51$ unequally-spaced grid points obtained from
\begin{equation}
\theta_i = \cos \left[ \pi \; (i-1) / (N-1) \right]
\;,\qquad
i = 1, \dots, N
\;.
\label{eq:prob2_gridpoints}
\end{equation}
Solve the resulting system using LU-decomposition, and compare results to the exact solution.

%%%%%%%%%%%%%%%%%%%%%%%%%%%%%%%%%%%%%%%%%%%%%%%%%
%%%%%%%%%%%%%%%%%%%%%%%%%%%%%%%%%%%%%%%%%%%%%%%%%
\section{Methodology} %%%%%%%%%%%%%%%%%%%%%%%%%%%
%%%%%%%%%%%%%%%%%%%%%%%%%%%%%%%%%%%%%%%%%%%%%%%%%
%%%%%%%%%%%%%%%%%%%%%%%%%%%%%%%%%%%%%%%%%%%%%%%%%

\subsection{Problem 1}

\subsection{Problem 2}

To discretize \eqref{eq:prob2_ode} using the unequally-spaced grid points given in \eqref{eq:prob2_gridpoints}, we use the central difference formula for a second-order accurate second-derivative,
\begin{equation}
\begin{aligned}
\frac{d^2 y}{d \theta^2}
&=
\left[
\frac{-2}{(\theta_{i+1} - \theta_{i-1})}
\left(
\frac{1}{\theta_{i+1} - \theta_i}
+
\frac{1}{\theta_i - \theta_{i-1}}
\right)
\right] y_i
\!\!\!\!
&+\left[
\frac{2}{(\theta_{i+1} - \theta_{i-1})(\theta_{i+1} - \theta_{i})}
\right] y_{i+1}
\\
&&+\left[
\frac{2}{(\theta_{i+1} - \theta_{i-1})(\theta_i - \theta_{i-1})}
\right] y_{i-1}
\\
&=
[D_i] \: y_i + [B_i] \: y_{i+1} + [C_i] \: y_{i-1}
\;,
\end{aligned}
\end{equation}
to rewrite the governing equation as
\begin{equation}
y'' + \frac{y}{4} = 0
\quad \rightarrow \quad
[B_i] \: y_{i+1} + [\tfrac{1}{4} + D_i] \: y_i + [C_i] \: y_{i-1} = 0
\;.
\end{equation}
Formation of the matrix equation proceeds in the standard manner for second-order central differences, resulting in a tridiagonal matrix equation $\mb{A} \mb{x} = \mb{f}$.

To solve our system for $\mb{x}$, we employ LU decomposition. The aim is to perform the decomposition $\mb{A} = \mb{L} \mb{U}$, where $\mb{L}$ is lower-triangular, and $\mb{U}$ is upper-triangular. Consider the tridiagonal matrix decomposition $\mb{A} = \mb{L} \mb{U}$ in more detail,
\begin{equation}
\begin{bmatrix}
a_1 & c_1 &        &         & \\
b_1 & a_2 & c_2    &         & \\
    & b_2 & \ddots & \ddots  & \\
    &     & \ddots & \ddots  & c_{N-1} \\
    &     &        & b_{N-1} & a_N
\end{bmatrix}
=
\begin{bmatrix}
l_1 &     &        &         & \\
b_1 & l_2 &        &         & \\
    & b_2 & \ddots &         & \\
    &     & \ddots & \ddots  & \\
    &     &        & b_{N-1} & l_N
\end{bmatrix}
\begin{bmatrix}
1 & u_1 &        &        & \\
  & 1   & u_2    &        & \\
  &     & \ddots & \ddots & \\
  &     &        & \ddots & u_{N-1} \\
  &     &        &        & 1
\end{bmatrix}
\;,
\end{equation}
where all blank elements are 0. The values of $l_i$ and $u_i$ are determined iteratively using
\begin{align}
l_1     &= a_1 \notag \\
u_{i-1} &= c_{i-1} \; / \; l_{i-1} \notag \\
l_i     &= a_i - b_{i-1} u_{i-1}
\;,
\qquad
i = 2, \dots, N
\;.
\end{align}

Back-substitution is used to find the solution vector $\mb{x}$. First, the system $\mb{L} \mb{z} = \mb{f}$ is solved using
\begin{align}
z_1 &= f_1 \; / \; l_1 \notag \\
z_i &= (f_i - b_{i-1} z_{i-1}) \; / \; l_i
\;,
\qquad
i = 2, \dots, N
\;.
\end{align}
Then we solve the system $\mb{U} \mb{x} = \mb{z}$ for $\mb{x}$ with
\begin{align}
x_N &= z_N \notag \\
x_i &= z_i - u_i x_{i+1}
\;,
\qquad
i = N-1, N-2, \dots, 1
\;.
\end{align}

LU decomposition is very efficient in scenarios where the governing equations remain the same, but solutions are desired for many instantiations of $\mb{f}$. Since $\mb{L}$ and $\mb{U}$ must be computed only once, repeated solution procedures are rapid once these matrices are known.

Finally, to assess numerical accuracy, we note that the analytical solution to \eqref{eq:prob2_ode} is
\begin{equation}
y(\theta) = \frac{2}{\sin(1)} \sin \left( \frac{\theta + 1}{2} \right)
\;.
\end{equation}

%%%%%%%%%%%%%%%%%%%%%%%%%%%%%%%%%%%%%%%%%%%%%%%%%
%%%%%%%%%%%%%%%%%%%%%%%%%%%%%%%%%%%%%%%%%%%%%%%%%
\section{Results} %%%%%%%%%%%%%%%%%%%%%%%%%%%%%%%
%%%%%%%%%%%%%%%%%%%%%%%%%%%%%%%%%%%%%%%%%%%%%%%%%
%%%%%%%%%%%%%%%%%%%%%%%%%%%%%%%%%%%%%%%%%%%%%%%%%

\subsection{Problem 1}

Analytical and central difference solutions to \eqref{eq:prob2_ode} are presented in \figref{fig:Prob2}. The global relative error, essentially summing over all points in the relative error plot below, is $1.61 \times 10^{-3}$.

\begin{figure}[h!]
\begin{center}
\includegraphics[height=1.95in]{Prob2_solution.eps}
\includegraphics[height=1.95in]{Prob2_error.eps}
\\[-0.5cm]
\caption{Analytical and central difference solutions to \eqref{eq:prob2_ode} using non-uniform mesh spacing, and the point-wise relative error.}
\label{fig:Prob2}
\end{center}
\end{figure}

\subsection{Problem 2}

%%%%%%%%%%%%%%%%%%%%%%%%%%%%%%%%%%%%%%%%%%%%%%%%%
%%%%%%%%%%%%%%%%%%%%%%%%%%%%%%%%%%%%%%%%%%%%%%%%%
\section{Discussion} %%%%%%%%%%%%%%%%%%%%%%%%%%%%
%%%%%%%%%%%%%%%%%%%%%%%%%%%%%%%%%%%%%%%%%%%%%%%%%
%%%%%%%%%%%%%%%%%%%%%%%%%%%%%%%%%%%%%%%%%%%%%%%%%

\subsection{Problem 1}

\subsection{Problem 2}

Results for the non-uniform grid solution match the exact analytical solution very well, as evinced by the low global and point-wise relative error. LU decomposition proves to be an effective method for solution of the matrix equation, but its superior efficiency is unable to be demonstrated in this problem, due to the fact that we are solving our matrix system once for a single RHS vector.

%%%%%%%%%%%%%%%%%%%%%%%%%%%%%%%%%%%%%%%%%%%%%%%%%
%%%%%%%%%%%%%%%%%%%%%%%%%%%%%%%%%%%%%%%%%%%%%%%%%
\section{References} %%%%%%%%%%%%%%%%%%%%%%%%%%%%
%%%%%%%%%%%%%%%%%%%%%%%%%%%%%%%%%%%%%%%%%%%%%%%%%
%%%%%%%%%%%%%%%%%%%%%%%%%%%%%%%%%%%%%%%%%%%%%%%%%

No external references were used other than the course notes for this assignment.

%%%%%%%%%%%%%%%%%%%%%%%%%%%%%%%%%%%%%%%%%%%%%%%%%
%%%%%%%%%%%%%%%%%%%%%%%%%%%%%%%%%%%%%%%%%%%%%%%%%
\section*{Appendix: MATLAB Code} %%%%%%%%%%%%%%%%
%%%%%%%%%%%%%%%%%%%%%%%%%%%%%%%%%%%%%%%%%%%%%%%%%
%%%%%%%%%%%%%%%%%%%%%%%%%%%%%%%%%%%%%%%%%%%%%%%%%

The following code listings generate all figures presented in this homework assignment.

%\includecode{Problem_1.m}
%\includecode{Problem_2.m}

%%
%% DOCUMENT END
%%
\end{document}
